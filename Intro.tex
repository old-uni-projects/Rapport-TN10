\chapter*{Introduction}
\markboth{Introduction}{}
\addcontentsline{toc}{chapter}{Introduction}

Dans le cadre de mes études en \emph{Génie Informatique, filière Fouille de Données (FDD)} à l'\textbf{Université de Technologie de Compiègne} (\url{http://www.utc.fr}), j'ai réalisé un stage Ingénieur de 6 mois dans l'entreprise DeltaCAD, spécialisée en informatique scientifique. 


Mon stage s'insère dans le cadre du projet de recherche, subventionné par l'ANR, METIS \\(\url{http://metis.deltacad.fr/}).

Au cours de mon stage, j'ai pu prendre part aux réunions de projet avec l'ensemble des partenaires académiques et industriels afin d'y présenter mes travaux.

L'objectif principal de ce stage fut de créer un "\textit{plugin de signature}" pour le logiciel METIS. Ce dernier devrait pouvoir signer une photo de pièce mécanique, et dans un deuxième temps être capable de comparer les signatures entres-elles.

Afin de mener cet objectif à bien, le stage fut composé de \textbf{différentes parties}.

Dans un premier temps, il fut primordial de réaliser une brève étude de l'\textit{état de l'art} associé au domaine de la rétro-conception de grands ensembles mécaniques.
Ensuite nous avons réorienté notre étude vers les méthodes existantes dans la littérature scientifique concernant la reconnaissance de forme ("\textit{Shape Matching}" et "\textit{Computer Vision}").

Nous avons pu identifier une méthode de mise en correspondance des formes grâce à l'utilisation de l'axe Median de Blum ou Squelette d'une Pièce. Cette méthode utilise également le concept de Shock Graph [K. Siddiqi et al]~\cite{Siddiqi1999}

Nous avons donc orienté mon travail sur la reprise du démonstrateur fonctionnel, mis en avant par [D. Macrini et al]~\cite{Macrini2002}, avec un objectif d'industrialisation du processus tout en apportant une vision orientée "\textit{grand volume de données}" au projet.
Dans un second temps il s'agira d'améliorer le résultat de l'algorithme de comparaison des \textit{ShockGraphs} et d'y adjoindre une fonctionnalité de préfiltrage des Graphs selon le contexte mécanique dans lequel on se place.

Finalement nous avons porté la solution vers le Cloud Amazon AWS, \url{http://aws.amazon.com/fr/}, afin d'assurer la fléxibilité en terme de stockage et puissance de calcul.

%\clearpage