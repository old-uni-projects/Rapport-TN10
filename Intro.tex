\chapter*{Introduction}
\markboth{Introduction}{}
\addcontentsline{toc}{chapter}{Introduction}

Dans le cadre de mes études en \emph{Génie Informatique} à l'\textbf{Université de Technologie de Compiègne} (\url{http://www.utc.fr}), j'ai réalisé un stage de 6 mois Assistant Ingénieur chez l'équipementier automobile Français : Valeo. J'ai travaillé à la direction des systèmes d'information du site d'Annemasse, dans le service informatique partagés entre plusieurs sites (Annemasse, Créteil, Abbeville, Mondeville, Ben Arous, Nevers). \\
Marc LAURENT, mon tuteur, est le manager de cette section.

Depuis déjà plusieurs années, Valeo s'est engagé dans une politique de standardisation de ses outils et méthodes dans le but d'améliorer les usages des nombreux sites et ainsi gagner en coût, délai et qualité.

C'est donc dans ce contexte de standardisation et que j'ai effectué mon stage, un environnement international marqué par la décentralisation des familles de produit Valeo. 

Mes activités ont été très variées au cours des 6 mois de stage. \\
J'y ai mené principalement deux missions: administrateur, formateur et développeur de la plateforme cloud de service \textit{Cordys Process Factory} (\url{http://www.cordys.com})  pour le périmètre cité plus haut, et finalement j'ai participé à la préparation du changement de version de l'ERP en place: \textbf{SAP}. Une mise à jour biaxiale portant sur la version logicielle et sur l'architecture de l'ERP qui se dirige vers un système unifié, hiérarchisé et centralisé pour tout Valeo. Le\textit{ Kick-Off }de ce projet devrait avoir lieu avant l'été 2014.

S'ajoutent à cela des tâches ponctuelles de support informatique, la participation à la réalisation d'un DRP sur l'ensemble des infrastructures informatiques du site Valeo de Nevers, j'ai également aidé à la bonne conduite d'un audit de sécurité sur le site Valeo d'Annemasse et finalement j'ai collaboré avec les équipes en charge de la réalisation d'une migration des lignes WAN et ISP en vue d'une augmentation du débit  tout en respectant la vision sécurité qu'apporte le DRP, une vision vitale pour l'entreprise. 

%\clearpage
