\section{ShapeLearner - Un plugin de signature}

METIS, en plus d'être un projet de recherche est également le nom du logiciel qui lui est associé. Ce dernier est programmé en JAVA. Il propose une interface à laquelle on peut venir adjoindre de multiples plugins de \textbf{signature}.

Ces derniers se doivent de répondre aux spécifications suivantes :
\begin{itemize}
	\item Être compilé sous forme d'une librairie DLL.
	\item Exposer une fonction de signature en contexte : \textit{Signature SignInContext(Context);}
	\item Exposer une fonction de comparaison en contexte : \textit{Output MatchInContext(Signature, Context);}
\end{itemize}

\subsection{Le plugin C++ ShapeLearner}

En se basant sur le travail effectué par D. Macrini et al.~\cite{Macrini2002}, nous avons décidé d'adapter le démonstrateur fonctionnel à nos besoins. En effet ce dernier a été publié sous licence Open Source, et a été entièrement programmé en C++.

Dans le soucis de faciliter l'intégration du démonstrateur à notre plugin, nous avons fais le choix d'utiliser nous aussi le langage C++.

\subsubsection{L'architecture logicielle}

\begin{itemize}
	\item \textbf{Module MM: }``Material Management" est le module logistique de SAP. Il gère les stocks, et les mouvements de stocks : entrées, sorties, et transferts.
	
	\item \textbf{Module PP: }``Production Planning" concerne la planification de la Production.
	
	 \item \textbf{Module SD: }``Sales and Distribution concerne l'administration des ventes.
	 
	 \item \textbf{Module FI: }``Financial contient toutes les écritures des ventes et achats, lesquelles se déversent dans la comptabilité générale via la comptabilité client ou fournisseur.
	 
	 \item \textbf{Module CO: }``Controlling concerne le contrôle de gestion.
	 
	 \item \textbf{Module BW: }``Business Information Warehouse est un outil de fouille de données et par extension de Reporting.
	 
\end{itemize} 

\clearpage

\subsection{Les intérêts Business de l'utilisation du système SAP}

De nombreuses entreprises utilisent de nos jours des systèmes d'information qui ont été mis en place pour exécuter un certains nombre de tâches spécifiques tout en apportant un outils de d'analyse et d'extraction des données transitant au sein du système.\\
Certains sont même capable de fournir des données temps-réels de l'état du système. En effet ces systèmes possèdent toutes les informations nécessaires en leur sein pour permettre une utilisation rapide et multi-support afin de gérer tout type d'évènements. 

Typiquement, une compagnie possédant de nombreux systèmes sous son contrôle, et différents processus critiques à gérer en parallèle comme la production, la vente, la comptabilité, ... aura un système d'information d'un grande complexité.  Chacun de ces systèmes fonctionnant avec une base de données bien spécifique. \\
Des interfaces de transfert de données entre ces systèmes sont utilisés aussi souvent que nécessaire. Le système devient très rapidement complexe et difficile à gérer. Le plantage d'un des éléments de tout le système d'information impliquera un service dégradé sur la globalité du système. Les coûts de maintenance, de sauvegarde ainsi que de maintient d'un DRP efficace deviennent prohibitifs.

SAP propose une vision différente, une vision unifiée: Un seul système d'information dans le système : SAP. 

Toutes les applications peuvent accéder aux données communes du système. Chaque événement qui a lieu dans le système est enregistré dans le système et immédiatement accessible par l'ensemble des modules. L'utilisation d'un système ERP permet une amélioration de la cohérence des données et de l'efficience du système. La simplification des actions, transactions et des processus réduit de plus la charge sur le système global. 

\clearpage