\section{Un système ERP : SAP}

%http://dspace.bracu.ac.bd/bitstream/handle/10361/1559/U2K2%20Project%20and%20SAP%20Execution.pdf?sequence=1
%

\subsection{SAP, vue d'ensemble}

SAP est un ERP (\emph{Enterprise Resource Planning} ou en français :  \textbf{Planification des Ressources de l'Entreprise}) développé par la société éponyme. Ce sigle signifie :\textbf{Systems, Applications, and Products for data processing}.

SAP est un puissant outil qui intègre de multiples Business Process et Fonctions en un seul système unifié. Fonctionnant sur une architecture Client / Serveur classique, chaque utilisateur (disposant des droits nécessaire) peut facilement gérer et tracer les ventes, la production, les finances de l'entreprise et les ressources humaines en temps réel. 

SAP est un système modulaire qui se compose de plusieurs parties. Chacune de ces parties peut elle même se diviser en plusieurs sous modules de fonction spécifique.
Tous les modules SAP sont cependant inter-connectés via une base de données centralisée qui connecte et recoupe les différents aspects et les différentes données de l'entreprise.

Pour n'en citer que quelques exemples, voici les quelques modules utilisés chez Valeo:

\begin{itemize}
	\item \textbf{Module MM: }``Material Management" est le module logistique de SAP. Il gère les stocks, et les mouvements de stocks : entrées, sorties, et transferts.
	
	\item \textbf{Module PP: }``Production Planning" concerne la planification de la Production.
	
	 \item \textbf{Module SD: }``Sales and Distribution concerne l'administration des ventes.
	 
	 \item \textbf{Module FI: }``Financial contient toutes les écritures des ventes et achats, lesquelles se déversent dans la comptabilité générale via la comptabilité client ou fournisseur.
	 
	 \item \textbf{Module CO: }``Controlling concerne le contrôle de gestion.
	 
	 \item \textbf{Module BW: }``Business Information Warehouse est un outil de fouille de données et par extension de Reporting.
	 
\end{itemize} 

\clearpage

\subsection{Les intérêts Business de l'utilisation du système SAP}

De nombreuses entreprises utilisent de nos jours des systèmes d'information qui ont été mis en place pour exécuter un certains nombre de tâches spécifiques tout en apportant un outils de d'analyse et d'extraction des données transitant au sein du système.\\
Certains sont même capable de fournir des données temps-réels de l'état du système. En effet ces systèmes possèdent toutes les informations nécessaires en leur sein pour permettre une utilisation rapide et multi-support afin de gérer tout type d'évènements. 

Typiquement, une compagnie possédant de nombreux systèmes sous son contrôle, et différents processus critiques à gérer en parallèle comme la production, la vente, la comptabilité, ... aura un système d'information d'un grande complexité.  Chacun de ces systèmes fonctionnant avec une base de données bien spécifique. \\
Des interfaces de transfert de données entre ces systèmes sont utilisés aussi souvent que nécessaire. Le système devient très rapidement complexe et difficile à gérer. Le plantage d'un des éléments de tout le système d'information impliquera un service dégradé sur la globalité du système. Les coûts de maintenance, de sauvegarde ainsi que de maintient d'un DRP efficace deviennent prohibitifs.

SAP propose une vision différente, une vision unifiée: Un seul système d'information dans le système : SAP. 

Toutes les applications peuvent accéder aux données communes du système. Chaque événement qui a lieu dans le système est enregistré dans le système et immédiatement accessible par l'ensemble des modules. L'utilisation d'un système ERP permet une amélioration de la cohérence des données et de l'efficience du système. La simplification des actions, transactions et des processus réduit de plus la charge sur le système global. 

\clearpage