\section{Un client = Une fonction partenaire}

\subsection{Les différentes fonctions partenaires}

Un client peut, classiquement, être référencé sous quatre formes dans un environnement professionnel.

\textbf{Ainsi nous y retrouvons:}

\begin{itemize}\itemsep7pt
	\item Le client \textbf{Donneur d'ordre}, fonction partenaire AG dans SAP (de l'Allemand: \textit{\textbf{A}uftrag\textbf{g}eber})
	\item Le client \textbf{Livré}, fonction partenaire WE dans SAP (de l'Allemand: \textit{\textbf{W}aren\textbf{e}mpfänger})
	\item Le client \textbf{Facturé}, fonction partenaire RE dans SAP (de l'Allemand: \textit{\textbf{R}echnungs\textbf{e}mpfänger})
	\item Le client \textbf{Payeur}, fonction partenaire RG dans SAP (de l'Allemand: \textit{\textbf{R}e\textbf{g}ulierer})
\end{itemize}

\subsubsection{Un exemple simple sur les fonctions partenaires}
Prenons une situation simple et courante pour expliquer que sont ces fonctions partenaires.

\textbf{Un enfant} veut acheter un cadeau à sa maman pour la fête des mères, pour cela il se connecte sur son site en ligne préféré et choisi ce qu'il veut commander.\\
Il passe la commande, seulement étant enfant il n'a pas de carte bancaire et ne peux donc payer ... \textbf{Papa} aide donc son fils à passer commande avec sa carte bancaire.\\
Une semaine plus tard, \textbf{Maman} reçoit le colis.

Dans cette histoire, l'\textbf{enfant} fait office de donneur d'ordre: C'est lui qui effectue la commande. \textbf{Papa} quand à lui grâce a sa carte bancaire est devenu le client payeur et le client facturé. Quand à \textbf{Maman}, elle est donc le client livré. 

Pour résumer:

\begin{itemize}\itemsep7pt
	\item Enfant = Client Donneur D'ordre = \textbf{AG}
	\item Papa = Client payeur et Client Facturé = \textbf{RE \& RG}
	\item Maman = Client Livré = \textbf{WE}
\end{itemize}
	
Cette explication est primordiale pour la suite car dans un contexte industriel il est très courant d'avoir quatre clients différents pour chacune des fonctions partenaires pour une seule et même commande.
	
Il est donc \textbf{nécessaire} d'établir une codification uniforme pour chacune de ces fonctions partenaires.

\clearpage

