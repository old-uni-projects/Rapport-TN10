\section{Titre de Fonction : ``Automation Analyst"}

\subsection{Description de mon rôle}

Mon rôle a été principalement orienté sur deux objectifs. En effet il m'a été demandé d'une part, de développer des améliorations sur un parc applicatif existant (en particulier par rapport aux attentes des ``Business Owners") ainsi que de développer et de déployer de nouvelles applications, d'autres part, il m'a été demandé d'effectuer le support applicatif sur ces mêmes applications via un système open source de tickets: GLPI.\\
Finalement, il m'a été demandé d'effectuer des formations \emph{orientées utilisateur} sur plusieurs sites Français à propos de l'utilisation des applications déployées en production.

Ce genre de poste demande donc une double compétence:

\begin{itemize}\itemsep7pt

	\item En effet, il nécessaire de connaître parfaitement son outil et la plateforme afin de développer la fonction qui manque ou d'améliorer l'interface d'une application. Une faculté d'analyse et d'adaptation rapide sont donc les qualités humaines prépondérantes dans cette activité en plus des connaissances techniques.
	
	\item Le deuxième rôle est d'effectuer un support relatif à une quinzaine d'applications. Il est donc extrêmement courant de d'interagir avec tout type d'utilisateur et ce plusieurs fois par jour. Le problème réside dans le fait que l'on ne connait que rarement son interlocuteur, tout comme sa fonction dans l'entreprise. Il faut donc être capable d'adapter son discours en fonction des connaissances de l'utilisateur, de son ``empressement" et de sa fonction dans l'entreprise. Il est donc critique d'avoir de bonnes facultés de communication (en anglais et français), une certaine capacité de discernement et de répartie.
	
\end{itemize} 

\clearpage

\subsection{Retour sur expérience}

\subsubsection{La partie technique}

Concernant la partie technique, Cordys Process Factory est très peu documenté. Impossible de trouver un manuel ou même une documentation officielle. Valeo possède ses propres modules de formation mais cela reste très succinct et ce fût très complexe de me former à  son utilisation, l'administration de la plateforme et particulièrement au développement sur cette plateforme. Malgré une formation intensive de deux jours à Paris, autonomie et des dizaines d'heures à tâtonner sur la plateforme furent les maitres mots de mon expérience sur Cordys Process Factory.

Un Forum Valeo pour les développeurs sur Cordys est en place, cet outil me fut d'un grand secours, surtout durant les premiers mois de mon stage.
Malgré cela je ne pense toujours pas maitriser la plateforme. Je connais les fonctions et les éléments qui me permettent de faire ce dont j'ai besoin, mais sur une demande trop spécifique ou trop pointue il m'est nécessaire de demander de l'aide. Cordys est une plateforme propriétaire de \textbf{développement au clic}, très peu de code est à mettre en place. Ce qui signifie que si vous ne savez pas où cliquer pour effectuer telle ou telle action, vous pouvez chercher pendant des heures avant que la solution vous apparaisse. A l'inverse d'un langage de programmation qui, pour la plupart, malgré ses spécificités, reste similaire à de nombreux autres et avec de solides connaissances en algorithmique et une documentation il est relativement aisé d'arriver à ses fins.

\subsubsection{La partie communicationnelle}

C'est sans aucun doute la partie qui m'a demandé le plus d'efforts. Tant dans la forme que dans le fond. En effet, je m'étais majoritairement adressé, auparavant, à des personnes connaissant l'informatique dans le cadre de mes études. Et ce ne fut pas forcément aisé d'adapter mon discours aux connaissances de mon interlocuteur et surtout au fait que je ne sache jamais à qui je m'adresse. Un environnement comme celui de Valeo est très complexe et demande de grandes précautions lorsque l'on s'adresse à quelqu'un. Il est donc primordial d'\textbf{apprendre} à ne s'engager sur quelque chose même si on le croit acquis et 100\% sûr. En effet, ce qui est possible dans un environnement quelconque n'est pas forcément possible dans l'environnement Valeo. A la fois d'un point de vue technique et également parce que l'on ne maitrise pas tous leviers d'une action et qu'il faut obtenir l'accord d'un certains nombre de personnes avant de pouvoir déclencher certaines actions.

J'ai personnellement eu beaucoup de mal à gérer la pression durant les deux premiers mois. En effet, un environnement comme Valeo demande que tout soit fait en flux tendu et en parallèle voir limite ``pour hier" pour ne paraphraser personne. C'est une situation à laquelle je n'avais jamais fait face, il a fallu \textbf{apprendre} à ne pas paniquer et se recentrer sur l'objectif, définir des priorités, gérer les timings et le tout de manière relativement automatique et intuitive car il n'est pas envisageable de passer plusieurs heures par semaine à établir un diagramme de Gantt pour la semaine. La \emph{liste des tâches} à effectuer de Google fut sans aucun doute mon meilleur allié pendant ce stage.
