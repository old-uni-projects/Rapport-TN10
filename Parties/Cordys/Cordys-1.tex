\section*{Avant Propos}

Cordys Process Factory fut un des points centraux de mon stage. Comme il ne m'est pas possible de décrire avec précision chacun de ses points, j'ai choisi de détailler mon travail dans la partie qui pour moi présente le plus d'intérêt. 

Cordys Process Factory étant une plateforme propriétaire et bénéficiant d'une interface de \emph{développement au clic}, je m'attacherai à en décrire avec minutie le fonctionnement, tous les outils qui sont mis à disposition sur la plateforme et leurs interactions avec autant que possible des références au travail que j'ai pu mené à Valeo.

\section{Une plateforme Applicative ``Cloud"}

\subsection{Qu'est ce que Cordys Process Factory ?}

Cordys Process Factory est un service orientée \textit{Cloud Computing}. Cette solution Cloud est de type PaaS : Plateforme as a Service. Elle permet de développer des applications répondant à une demande de \textit{``Business Workflow"}. C'est un besoin critique dans un environnement comme celui de Valeo. 

Les utilisateurs classiques, tout comme les développeurs, peuvent utiliser la plateforme pour \textbf{composer} des applications de manière simple et rapide en ayant reçu une brève formation sur l'outil(\textbf{sans connaissances en programmation informatique} ou presque). \\
La plateforme est également capable de s'interfacer avec les différents systèmes existants et les \textit{Cloud Services} actuellement en fonction dans l'entreprise. Un point qui fut très important dans le choix de cette plateforme.

\clearpage

\subsection{Lotus Notes vs Cordys Process Factory}

Lotus Notes (solution logicielle développée par \textbf{IBM}) et Cordys Process Factory (solution Cloud développée \textbf{Open~Text}) ont un objectif commun : proposer une réponse au  besoin de ``\textit{Business Processing}" au sein d'une entreprise.

Cependant elles diffèrent sur l'architecture (serveurs et réseau) nécessaire à son fonctionnement :

\begin{itemize}
		\item Lotus Notes se base sur une architecture type logiciel Client / logiciel Serveur, les serveurs sont inter-connectés entre eux et ``\textit{répliquent}" certaines de leur données à intervalle de temps régulier. Ce fonctionnement est coûteux , il demande une infrastructure imposante ainsi qu'une administration au jour le jour. \\
		Plus gênant, comme il est difficile d'avoir une vue d'ensemble du système actuel, la quantité d'application dupliquée ne cesse d'augmenter, ce qui induit des coûts de fonctionnement de plus en plus élevés.\\

	\item Cordys Process Factory propose une réponse orientée ``\textit{Cloud Computing}" à ces problématiques, nous pouvons donc faire \textit{abstraction} de l'architecture réseau et serveur nécessaire à son fonctionnement (gérée par le fournisseur du service). La plateforme est accessible via un navigateur web, nul besoin de paramétrer son ordinateur ou d'installer un quelconque logiciel.\\
	Le choix de Cordys Process Factory est donc cohérent avec la politique actuelle du groupe Valeo qui cherche à \emph{externaliser} la gestion de la partie infrastructure et l'administration des diverses solutions informatiques.\\
	\end{itemize}


\subsection{Le contexte dans lequel s'insère Cordys Process Factory}

Valeo a fait le choix de se baser en grande partie sur la suite Google Apps pour un maximum de services et besoins bureautiques. Ainsi nous utilisons de manière non exclusive: 

	\begin{itemize}
		\item Gmail
		\item Google Drive
		\item Google Apps (SpreadSheet, Docs, Presentation, Scripts, Sites)
		\item Google Agenda 
		\item Google AppEngine
	\end{itemize}

En complément nous utilisons un LDAP, commun à tous les sites Valeo du monde, interfacé avec la suite applicative de \textbf{Google}. Valeo appelle cet ensemble de systèmes inter-opérants  : VeGA \textit{(Valeo empowered by Google Apps)}.

\clearpage

\textbf{Et \textit{Cordys Process Factory} dans tout ça ?}

Fort de sa position de partenaire \emph{Google Entreprise}, Cordys Process Factory permet une inter-connection aux services Google Apps  et le LDAP de Valeo : \textit{l'Entreprise Directory}.\\
Ainsi les \textit{Businness Process} et \textit{Businness Workflows} peuvent s'appuyer sur les services et applications déjà en place chez Valeo afin de réduire les coûts de mise en place et de développement.

\clearpage