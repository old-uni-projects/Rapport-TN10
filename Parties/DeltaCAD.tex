\chapter{Présentation de l'entreprise}

\section{DeltaCAD, une entreprise d'informatique scientifique}

\subsection{Présentation de l'entreprise}

\textbf{DeltaCAD} est une société d'ingénierie spécialisée en CAO/Simulation et en informatique scientifique, créée en 1994.

Ses ingénieurs ont une expérience de plus de 20 ans dans ce domaine. Grâce à ses compétences reconnues par le marché, elle propose des services et des logiciels pour aider ses clients à concevoir, dimensionner, optimiser leurs produits, systèmes et procédés de fabrication.

L'offre de \textbf{DeltaCAD} est orientée vers les entreprises à fort potentiel technologique dans les secteurs:\\
\begin{itemize}
  \item Automobile
  \item Électronique
  \item Mécanique
  \item Énergie
  \item Aéronautique
  \item Génie Civil
  \item Édition de Logiciels
\end{itemize}
\vspace{5mm}

\textbf{DeltaCAD} propose, en outre, à ses clients et partenaires industriels :\\
\begin{itemize}
  \item Des services d'études, de conseil, de développement de logiciels et de gestion globale de logiciels spécifiques
  \item Des produits logiciels dans le domaine de la géométrie, du maillage et du transfert d'informations entre logiciels de modélisation
\end{itemize}
\vspace{5mm}


\subsection{Un statut de SCOP}

\textbf{DeltaCAD} est une \textbf{S}ociété  \textbf{CO}opérative et \textbf{P}articipative (SCOP). C'est un statut spécifique qui stipule que les salariés sont associés majoritaires et détiennent au moins 51\% du capital social et 65\% des droits de vote. Si tous les salariés ne sont pas associés, tous ont vocation à le devenir.\\
On y retrouve, un dirigeant élu par les salariés associés.

Le profit de l'entreprise est équitablement partagé entre les salariés.
\begin{itemize}
  \item Une part pour tous les salariés, sous forme de participation et d’intéressement
  \item Une part pour les salariés associés sous forme de dividendes
  \item Une part pour les réserves de l’entreprise 
\end{itemize}
\vspace{5mm}

\textbf{Source : \url{http://www.les-scop.coop/sites/fr/les-scop/qu-est-ce-qu-une-scop.html}}

\subsection{L'offre de services}

Pour répondre aux différents besoins des métiers de ses clients, DeltaCAD propose tous les services nécessaires à l'ensemble du cycle de vie des applications scientifiques et techniques de leurs clients et partenaires industriels

Tous ces services s'appuient sur les compétences fortes de ses ingénieurs et les nombreux projets que DeltaCAD a menés avec succès pour ses clients depuis plus de 20 ans.

\textbf{Source : \url{http://www.deltacad.fr/}}, rubrique \textit{Services}

\subsubsection{Expertise et conseil en systèmes CAO/Simulation}

\textbf{DeltaCAD} apporte son expertise pour répondre à des questions stratégiques d'évolutions de logiciels ou d'organisation que se posent ses clients. Ces questions recouvrent des aspects variés : \\
\begin{itemize}
  \item Audits de logiciels (architecture, évolutivité, portabilité, performances...)
  \item Choix de composants logiciels stratégiques
  \item Validation externe de logiciel de simulation (taux de couverture, robustesse, qualité des résultats...)
  \item Plan Qualité logiciel
  \item Choix et méthodologie d'utilisation des logiciels de CAO/Simulation
  \item Capitalisation du savoir-faire métier
  \item Rédaction de cahier des charges logiciels
\end{itemize}

\subsubsection{Développement d'applications scientifiques et techniques}

De par son expérience des différents métiers et techniques, \textbf{DeltaCAD} développe des logiciels solutions permettant de répondre aux besoins spécifiques de ses clients. \textbf{DeltaCAD} maîtrise, en partenariat avec son client, la réalisation complète de l'application de sa conception à sa maintenance.\\

\subsubsection{Intégration et couplage de logiciels de CAO/Simulation (application métier)}

De par son expérience des différents métiers et techniques, \textbf{DeltaCAD} développe des logiciels solutions permettant de répondre aux besoins spécifiques de ses clients. \textbf{DeltaCAD} maîtrise, en partenariat avec son client, la réalisation complète de l'application de sa conception à sa maintenance.\\

\clearpage
\subsubsection{Industrialisation et gestion déléguée de logiciels}

Les entreprises clientes de \textbf{DeltaCAD} réalisent en interne des applications représentant un savoir-faire stratégique pour leurs métiers.

\textbf{DeltaCAD} a développé les méthodes et l'organisation nécessaires pour intervenir sur tout ou partie du cycle de vie des logiciels métiers. Ceci garantit une continuité de l'évolution et du support de l'application avec une totale visibilité pour l'entreprise cliente, tout en lui  permettant de se concentrer sur son métier.

\textbf{DeltaCAD} met ainsi à la disposition de ses clients des moyens analogues à ceux qui contribuent au succès de ses produits.

\subsubsection{Etudes, calculs de produits et processus}

La maîtrise de la simulation de nombreux phénomènes physiques (Mécanique, Thermique, Fluide) et la réactivité de DeltaCAD leur permettent de réaliser des simulations adaptées à de nombreux contextes.

Pour garantir la réactivité et la qualité des résultats, \textbf{DeltaCAD} dispose de moyens de calculs et de logiciels performants, et d'une expérience de plus d'une vingtaine d'années en modélisation.

Ces études peuvent également être réalisées avec des logiciels spécifiques développés par \textbf{DeltaCAD} pour les besoins particuliers de l'étude.

\clearpage
