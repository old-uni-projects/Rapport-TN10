\chapter{Présentation du Sujet de Stage}

	\textbf{Titre du stage :}
	
	Implémentation d’algorithmes de reconnaissances de formes mécaniques dans plusieurs nuages de points issus de numérisation.
	
	\textbf{Description :}
	
Dans le cadre d’un laboratoire commun entre la société DeltaCAD et l’université de Technologie de Compiègne nommé \textbf{DiMEXP} (\url{https://dimexp.utc.fr/}) – \textit{\textbf{Di}gital \textbf{M}ock-up for \textbf{M}ulti-\textbf{EXP}ertise Integration}, une équipe de chercheurs et ingénieurs R\&D travaillent sur l’intégration de données hétérogènes au sein de maquettes numériques de grands ensembles. 

Il s'agira d'intervenir sur de nouveaux modules et fonctionalités pour les suites logicielles développées par \textbf{DeltaCAD} qui visent le domaine de la préparation de maquettes numériques dans les différentes étapes du cycle de vie des produits (par exemple pour faciliter l’utilisation de gros modèles 3D en aéronautique, nautique, automobile, ...). 

Il faudra notamment, développer et implémenter des algorithmes ciblant la recherche de composants mécaniques dans un ensemble de données de nature hétérogène comme les nuages de points issus de scan 3D, des images issues de photogrammétries et des modèles CAO. Le concept de \textit{signature de composant mécaniques} se positionne comme élément central du développement. 

Dans le cadre de DIMEXP, quelques signatures ont pu déjà être mise au point basées sur des facteurs tel que la compacité, ou les \textit{reeb–graphs} ou encore les \textit{precedence-graphs}. La finalité est de pouvoir tester ces signatures et obtenir une reconnaissance statistique des formes mécaniques (de type automobile et aéronautique) sur des données 2D et 3D de grande dimension. Cette reconnaissance permettra \textit{in fine} d’élaborer une nomenclature (nombre d’apparition dans les nuages de points et maquette numérique du composant). La robustesse et les temps de calculs seront également à prendre en compte pour l’implémentation.

Une attention particulière sera apportée à la construction d'une base de connaissances à partir de données hétérogènes fournies (photographies, imagerie 3D, nuages de points, modèle CAO, ...). Cette base de connaissances vous servira pour élaborer un algorithme d'apprentissage supervisé de reconnaissance d'objets mécaniques.

\clearpage
