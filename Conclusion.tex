\chapter{Conclusion}

C'est avec plaisir que maintenant je peux regarder en arrière, rire de mes erreurs, sourire de mes nombreuses questions mais surtout être satisfait du chemin parcouru.

Les nombreux problèmes qui se sont posés à moi m'ont forcé à toujours aller plus loin dans mes raisonnements et mes recherches, à trouver \textit{la parade} ou \textit{l'astuce} qui fera que \textit{cette fois-ci, ça marchera}.\\
Un moment où j'ai pu tiré profit de l'enseignement reçu à l'UTC et en Erasmus à Vienne et mettre en difficulté mes connaissances afin de consolider mes acquis.\\
En particulier dans le cadre de mon travail sur SAP, j'ai eu la chance de créer un outil informatique qui sera par la suite réutilisé et peut être améliorer par les équipes Valeo ultérieurement.

J'en profite, encore une fois, pour remercier tout ceux qui ont participé au bon déroulement de ce stage qui m'aura permis d'avoir un vaste aperçu du monde de l'entreprise, de me familiariser avec un environnement complexe comme celui de Valeo, de découvrir des sujets passionnants. Ce stage fut l'occasion de confronter des connaissances théoriques à des connaissances pratiques et au monde de l'entreprise.

Je ne saurais faire la liste de tout ce que j'ai pu apprendre et j'aurais souhaité que ce stage dure un peu plus longtemps pour avoir le temps de me pencher sur le logiciel BW, module SAP de Fouille de données.

Ce stage m'a donc beaucoup appris dans des domaines techniques et concernant la vie en entreprise, et cette expérience me sera très utile pour m'adapter dans différents types d'organisations. Il est un premier élan dans le monde du travail en tant qu'ingénieur. Une première vraie prise de responsabilité professionnelle que m'a offerte Valeo.