\chapter{Conclusion}

C'est avec plaisir que, maintenant, je peux regarder en arrière, rire de mes erreurs, sourire de mes nombreuses questions mais surtout être satisfait du chemin parcouru.

Les nombreux problèmes qui se sont posés à moi m'ont forcé à toujours aller plus loin dans mes raisonnements et mes recherches, à trouver \textit{la parade} ou \textit{l'astuce} qui fera que \textit{cette fois-ci, ça marchera}.

Ce fut également un moment où j'ai pu tirer profit de l'enseignement reçu à l'UTC ainsi que lors de mes deux Erasmus à Vienne et Hamburg. Ce fut également l'occasion de mettre en difficulté mes connaissances afin de consolider mes acquis.

J'ai eu l'occasion d'acquérir une première expérience du monde de la recherche avec un projet concret : METIS. Mais également l'opportunité de découvrir un cas concret d'utilisation de la Vision par Ordinateur et de l'apprentissage automatique. Ce fut également une chance unique de découvrir le monde du développement logiciel et le développement de services basés sur le cloud mais également d'entrevoir les problématiques et challenges de la rétro-conception de grands ensembles mécaniques.

J'en profite, encore une fois, pour remercier tous ceux qui ont participé au bon déroulement de ce stage. 

Je ne saurais faire la liste de tout ce que j'ai pu apprendre, cette expérience me sera très utile pour la suite de mon projet professionnel: Réaliser une thèse sous la direction de M. Durupt à l'UTC et la codirection de M. Kiritsis à l'EPFL. Il est un premier élan dans le monde de la recherche et un point final à mon cursus d'ingénieur.