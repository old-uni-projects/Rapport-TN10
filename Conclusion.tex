\chapter{Conclusion}

C'est avec plaisir que maintenant je peux regarder en arrière, rire de mes erreurs, sourire de mes nombreuses questions mais surtout être satisfait du chemin parcouru.

Les nombreux problèmes qui se sont posés à moi m'ont forcé à toujours aller plus loin dans mes raisonnements et mes recherches, à trouver \textit{la parade} ou \textit{l'astuce} qui fera que \textit{cette fois-ci, ça marchera}.\\

Un moment où j'ai pu tiré profit de l'enseignement reçu à l'UTC ainsi que pendant mes deux Erasmus à Vienne et à Hamburg. Ce fut également l'occasion de mettre en difficulté mes connaissances afin de consolider mes acquis.\\

J'ai eu l'occasion d'acquérir une première expérience du monde de la recherche avec un projet concret : METIS. Mais également l'opportunité de découvrir un cas concret d'utilisation de la Vision par Ordinateur et de l'apprentissage automatique. Ce fut également une occasion unique de découvrir le monde du développement logiciel et le développement de services basés sur le cloud.

J'en profite, encore une fois, pour remercier tout ceux qui ont participé au bon déroulement de ce stage. Ce dernier m'aura permis d'acquérir un aperçu du monde de l'entreprise, de la recherche et de la rétro-conception mécanique mais également de confronter des connaissances théoriques à des connaissances pratiques, au monde de l'entreprise et de la recherche.

Je ne saurais faire la liste de tout ce que j'ai pu apprendre, cette expérience me sera très utile pour la suite de mon projet professionnel: Réaliser une thèse sous la direction de M. Durupt à l'UTC et la codirection de M. Kiritsis à l'EPFL. Il est un premier élan dans le monde de la recherche et un point final à mon cursus d'ingénieur.